\documentclass[12pt]{article}
\usepackage[utf8]{inputenc}
\usepackage{amssymb}
\usepackage{amsmath}
\newcommand{\Z}{\mathbb{Z}}
\newcommand{\R}{\mathbb{R}}
\newcommand{\hk}{\widehat{h}_k}
\newtheorem{theorem}{Theorem}[section]
\newtheorem{corollary}{Corollary}[theorem]
\newtheorem{lemma}[theorem]{Lemma}
\setlength\parindent{0pt}
\usepackage{geometry}
\geometry{legalpaper, margin=1in}
\title{Derivative of Das-Sander-Yorke Algorithm}
\author{Jieun Seong}
\date{Last edited on \today}

\begin{document}

%\maketitle

Goal: show that the sum and the derivative for DSY algorithm commute by proving that the sum of derivatives converges.

\vfill

By Fourier, $h(\theta)=\sum_{k\in\Z}\hk e^{2\pi i k \theta}$, so $Dh(\theta)=\sum_{k\in\Z} \hk 2\pi i k e^{2\pi i k \theta}$. Let $\theta = h^{-1}(y)$. 

\begin{equation*}\begin{split}
     &\frac{1}{A_L} \sum_{N=1}^L w\left(\frac{N}{L}\right) Dh \circ R^N \circ h^{-1}(y) \sum_{j=0}^{N-1} (Dh^{-1} \circ h \Dot{f} \circ f^{-1} \circ h) \circ R^j (\theta)\\
     &= \frac{1}{A_L}\sum_{N=1}^L w\left(\frac{N}{L}\right) Dh(\theta+\omega N) \sum_{j=0}^{N-1} (Dh^{-1} \circ h \Dot{f} \circ f^{-1} \circ h) \circ R^j (\theta)\\
     &= \frac{1}{A_L}\sum_{N=1}^L w\left(\frac{N}{L}\right) \sum_{k\in\Z}\hk 2 \pi i k e^{2\pi i k (\theta+\omega N)} \sum_{j=0}^{N-1} (Dh^{-1} \circ h \Dot{f} \circ f^{-1} \circ h) \circ R^j (\theta)\\
     &\text{Let $s=\frac{t}{L}$. Then, $Lds=dt$, $t=sL$. By Poisson Lemma,}\\
     &= \frac{1}{A_L}\sum_{N=1}^L \sum_{k\in\Z}\hk 2 \pi i k e^{2\pi i k \theta} \int_\R w(s)e^{2 \pi i (k\omega - N) sL}Lds \sum_{j=0}^{N-1} (Dh^{-1} \circ h \Dot{f} \circ f^{-1} \circ h) \circ R^j (\theta)\\
     &=\frac{1}{A_L} \sum_{k\in\Z}\hk 2 \pi i k e^{2\pi i k \theta} \sum_{N=1}^L \int_\R w(s)e^{2 \pi i (k\omega - N) sL}Lds \sum_{j=0}^{N-1} (Dh^{-1} \circ h \Dot{f} \circ f^{-1} \circ h) \circ R^j (\theta)\\
     &\text{Let $\gamma = 2\pi(k\omega-N)L$.} \\
     &\text{Integrating by parts, $\left| \int_\R w(s)e^{i\gamma s} \right| = \left| \frac{1}{\gamma^m} \int_0^1 w^{(m)}(s)e^{i\gamma s}ds\right| \leq |\gamma|^{-m}\|w^{(m)}\|_1. $}\\
     &\text{Also, $\left|\sum_{j=0}^{N-1} (Dh^{-1} \circ h \Dot{f} \circ f^{-1} \circ h) \circ R^j (\theta)\right|\leq aN$, for some constant $a$. Thus,}\\
     &\left| \frac{1}{A_L} \sum_{N=1}^L w\left(\frac{N}{L}\right) Dh \circ R^N \circ h^{-1}(y) \sum_{j=0}^{N-1} (Dh^{-1} \circ h \Dot{f} \circ f^{-1} \circ h) \circ R^j (\theta) \right| \\
     &\leq \frac{L}{A_L} \sum_{k\in\Z} \left| \hk 2\pi i k e^{2\pi i k \theta} \right| \sum_{N=1}^L|\gamma|^{-m}\|w^{(m)}\|_1 aN\\
     &\text{Note that $\frac{A_L}{L} \rightarrow \int_0^1 w(s)ds \neq 0$ as $L\rightarrow\infty$, so $\frac{L}{A_L} \leq C_w$ for some constant $C_w$. }\\
     &\text{Plug back in $\gamma$.}\\
     &\leq 2\pi a C_w \|w^{(m)}\|_1 \sum_{k\in\Z} |\hk||k| \sum_{N=1}^L|2\pi(k\omega-N)L|^{-m} N\\
     &=2\pi a C_w \|w^{(m)}\|_1 (2\pi L)^{-m} \sum_{k\in\Z} |\hk||k| \sum_{N=1}^L|k\omega-N|^{-m} N\\
     &\text{Let $\Tilde{C}_w = 2\pi a C_w \|w^{(m)}\|_1 (2\pi)^{-m}$. }\\
     &=\Tilde{C}_w L^{-m} \sum_{k\in\Z} |\hk||k| \sum_{N=1}^L|k\omega-N|^{-m} N\\
     &\text{Note that $\sum|k\omega-N|^{-m}N\leq\sum\left|k\omega-N\right|^{-m}\left(|N-k\omega|+|k\omega|\right) 
     \leq C_m(1+|k\omega|)$ for some $C_m$.}\\
     &\leq\Tilde{C}_wC_m L^{-m} \sum_k |\hk||k|(1+|k\omega|)\\
     &\text{Remark. Finite regularity is enough. $|\hk|\leq C_r |k|^{-r}\|h\|_{C^r}$ for some constant $C_r$.}\\
     &\leq \Tilde{C}_wC_m L^{-m} \sum_k C_r|k|^{-r}\|h\|_{C^r}|k|(1+|k\omega|)\\
     &=\Tilde{C}_wC_m C_r L^{-m} \|h\|_{C^r} \sum_k |k|^{-r+1}(1+|k\omega|)
\end{split}\end{equation*}

This will converge if $r>3$. Q.E.D.

\end{document}